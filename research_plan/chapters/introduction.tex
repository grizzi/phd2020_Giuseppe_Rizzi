\section{Introduction}
\label{sec:Introduction}


\begin{itemize}
	\item \emph{What is this research plan about?}
\end{itemize}

There is growing interest in deploying autonomous systems in every-day life. Robots have the power to enhance productivity while relieving human labor from repetitive and dangerous tasks. In many applications, these tasks consist of a physical interaction between the autonomous agent and the surrounding environment. Opening a door, pushing a button, turning a switch are just a few of the recurring interaction instances. We refer to the autonomous execution of them with the umbrella term of \emph{autonomous manipulation}. Complex manipulation is demanded in assistive service in the healthcare domain~\cite{cooper2020ari} , agriffods~\cite{duckett2018agricultural} and industrial inspection and maintainance~\cite{lattanzi2017review}. This research proposal contributes to advance the state of the art of autonomous manipulation as part of the Horizon 2020 framework and Piloting European project~\cite{eu-piloting-2020}. Recent successful applications (cite something) in search and rescue have shown that robots are a priceless alternative to life threatening human operations. With this motivation, the project aims to successfully deploy ground and aerial unmanned vehicles for industrial maintainance and inspection. 

\begin{itemize}
    \item \emph{What are the (high-level) research gaps?}
\end{itemize}

The complex manipulation task is either decomposed in smaller tasks or solved end to end. (expand)

Limitation of the usual framework where the task is decomposed are: the modules are not aware of each other, too rigid framework not the way human beings work (ref allowances), does not account for uncertainty into the environment (ref active perception) (expand). 

Limitation of the end-to-end approach is that they do not scale well, impractability of using such methods reliably on the real hardware, poor efficiency (expand).

\begin{itemize}
    \item \emph{What is the overall goal?}
\end{itemize}

The overall goal is to come up with a tight perception and control framework which is able to address robustly the problem of autonomous robotic manipulation alleviating the limitation of the current approaches and providing scalability, applicability to the real platform while staying as general as possible.
