\section{Introduction}
\label{sec:Introduction}


\marginpar{\emph{What is this research plan about?}}

There is growing interest in deploying autonomous systems in everyday life. Robots have the power to enhance productivity while relieving human labor from repetitive and dangerous tasks. In many applications, these tasks consist of a physical interaction between the autonomous agent and the surrounding environment. Opening a door, pushing a button, turning a switch are just a few of the possible interactions. We refer to the autonomous execution of them with the umbrella term of \emph{autonomous manipulation}.

Complex manipulation is demanded in assistive service in the healthcare domain~\cite{cooper2020ari}, agrifoods~\cite{duckett2018agricultural} and industrial inspection and maintenance~\cite{lattanzi2017review}. The proposed research aims to advance the state of the art in autonomous manipulation as part of the EU Horizon 2020 project Piloting~\cite{eu-piloting-2020}. Recent successful applications in search and rescue have shown that robots are a priceless alternative to life threatening human operations. With this motivation, the project's consortium aims to successfully deploy ground and aerial unmanned vehicles for industrial maintenance and inspection. 

\marginpar{\emph{What are the (high-level) research gaps?}}

Manipulation is hard. This research field encompasses hardware and software design. The former focuses on finding new layouts, morphologies and actuation principles that bring robots closer to the human dexterous and mobility capabilities. In this research proposal, the goal is to investigate new and more effective algorithmic solutions to manipulation on readily available robotic platforms. Complexity in robotic manipulation arises from the combination of multiple research sub-topics, namely \emph{perception}, \emph{modeling} and \emph{control}. The preeminent approach found in the literature consists of addressing these aspects separately. We claim that in order to push further the boundaries of autonomous manipulation, a joint effort is needed. 

\paragraph{Perception} The real world is too complex to expect an accurate model of the environment and the objects which are contained in it. For this reason we need to perceive and understand every new scene. Autonomous manipulation requires a denser and interaction-rich information which is not provided by pure passive observation. We need perception to provide a broad appreciation of the scene but also to provide high-resolution information, including knowledge of the contact locations and forces exchanged~\cite{mason2018toward}. Nevertheless, perception is often treated separately from control. As a consequence, perception systems are developed whose representation of the world is not optimal from the control perspective. In the last decade, perception systems have focused on tasks such as classification~\citep{redmon2016you}, semantic segmentation~\cite{badrinarayanan2017segnet}, generative modeling~\citep{karras2019stylebased} and pose estimation~\cite{xiang2017posecnn}. 

\paragraph{Control and Planning} We can define manipulation as the act of changing the state of the environment (object) by exchanging interaction forces with it. Interaction consists of point-to-point, point-to-plane, continuous and discontinuous contacts. Furthermore, exchanged wrenches could be high under incorrect modeling assumptions and may damage the robot and the surrounding environment. In order to avoid high interaction wrenches with a rigid environment we need a compliant control strategy which can reason about the scene physics. This is a complex task which is generally decoupled into a separate planning and control stages.  


\paragraph{Model Uncertainty} The geometry of the scene can be measured only at the accuracy allowed by the visual sensors and perception pipeline. The modeling of the environment and its physical properties can also be inaccurate. Consider the apparently easy task of turning a crank. The required motion is a simple circle. The problem is, where exactly should the robot produce the circle? We can do our best to estimate the crank’s position, but we will never get it exactly right \citep{mason2018toward}. Uncertainty is often treated as a metric in the state estimation pipeline rather than a variable to actively account for during control. So generally, control validation is performed with ground-truth information or the architecture is designed such that it complies with a small degree of uncertainty. Yet, do humans count on perfect knowledge of the environment for interaction? We observe that we do not but we are still able to perform myriad complex interaction tasks. We must therefore deduce that environment knowledge is improved ``on-the-fly" and we use the unconscious knowledge of uncertainty to perceive, plan and control as a whole. As an example, what would a human do when trying to turn on a light switch in the dark? She would probably reach the visible wall next to the door and follow the surface until sensing a switch-shaped object. We need perception and control algorithms that actively take uncertainty into consideration in order to achieve robust manipulation capabilities under hard sensing conditions.   
\medskip 
\newline
We have found the main gaps in the state of the art to be:
\begin{itemize}
\item \textbf{Suboptimal scene representation} from visual inputs.
\item \textbf{Rigid plan-and-act control} architecture
\item \textbf{Lack of reasoning about uncertainty} in perception and modeling. 
\end{itemize}

\marginpar{\emph{What is the overall goal?}}

The overall goal of this thesis is to develop a tight perception and control framework which is able to robustly address the problem of autonomous robotic manipulation of articulated objects alleviating the limitation of the current approaches and providing scalability, generalizability and applicability to the real platform. In the following sections we are going to expand over the related work focusing on the aspects that we want to address in this research proposal. Thereafter the approach is presented. The main objective is subdivided into simpler sub-tasks which are grounded on clear research hypotheses. The research plan is concluded with a list of proposed publications and the accompanying time plan forecast to achieve the intended goals. 
