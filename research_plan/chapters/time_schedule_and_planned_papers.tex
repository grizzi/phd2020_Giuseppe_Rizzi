\section{Time Schedule and Planned Publications}
\label{sec:time_schedule}

\marginpar{
\begin{itemize}
   \item \emph{How can each topic be subdivided into meaningful tasks?}
	\item \emph{What is the time schedule?}
	\item \emph{Which publications are planned (list of planned publications)?}
\end{itemize}
}

\subsection{Planned Publications}
\paragraph{Stochastic Control for Reactive Whole-Body Manipulation} We demonstrate the applicability of sampling-based method to whole-body coordination for the task of opening a cabinet door. We provide an optimized open-source which is general and can be used to solve a general set of robotic tasks. We introduce system and algorithmic adaptations that enable use to successfully transfer the algorithm on the real robot.

\paragraph{Affordace Based Control} We investigate how to use affordances for control. Following the research thread about dense affordace predictions, we use these to create a cost map to be used by the sampling based controller to drive the manipulator towards interaction points thus removing the need for highly engineered cost functions and interaction priors. We evaluate the approach in simulation and on the real system. 

\paragraph{Interactive Affordances} We enrich pure geoemetrical affordances with wrench information. We will develop a wrench interaction scheme to provide supervision to a learning agent about successful interaction points, orientations and wrenches. The method will be evaluated on a simulated dataset and on real data of various articulated objects such as cabinet doors, drawers, switches, and handwheel valves. Data sources consists of RGB-D images and point clouds. 


\paragraph{Interactive Perception} We investigate how to levarage the previous work to perform manipulation robust to modeling uncertainty. We rely on off-the-shelf methods to get a initial coarse estimate of the articulated model and its parameters. This initial estimate is fed to the estimation and control pipeline as a prior which is then refined in real time through interaction, while performing the task. 

The detailed time plan is shown in the Gantt chart shown in \cref{fig:Gantt Chart}.

% http://www.ctan.org/pkg/pgfgantt

\begin{figure}[]
\begin{center}

\begin{ganttchart}[
x unit = 0.45cm,
y unit title=0.4cm,
y unit chart=0.45cm,
vgrid,
hgrid,
title label anchor/.style={below=-1.7ex},
title height=1,
bar/.style={fill=gray!50,
            draw=black,
            line width=0.5pt},
incomplete/.style={fill=white},
progress label text={},
bar height=0.7,
group left shift=0.0,
group right shift=0.0,
group top shift=0.5,
group height=0.2,
group peaks width=0.4,
group peaks height=0.25,
group peaks tip position=0,
milestone height=0.7,
milestone label anchor/.style={below left=-1.7ex and 0.2ex}
]{1}{24}

% Labels
\gantttitle{Month}{24} \\
\gantttitlelist{1,...,24}{1} \\

% Group 1
\ganttgroup{Group 1}{1}{12} \\
\ganttbar{Task 1}{1}{2} \\
\ganttbar{Task 2}{3}{8} \\
\ganttbar{Task 3}{9}{10} \\
\ganttbar{Task 4}{11}{12} \\
\ganttmilestone{Publication 1}{12} \\

% Group 2
\ganttnewline
\ganttgroup{Group 2}{13}{24} \\
\ganttbar{Task 5}{13}{16} \\
\ganttbar{Task 6}{15}{18} \\
\ganttbar{Task 7}{19}{22} \\
\ganttmilestone{Publication 2}{22} \\
\ganttbar{Task 8}{23}{24}

% Relations
\ganttlink{elem3}{elem7}

\end{ganttchart}
\end{center}
\caption{Gantt Chart}
\label{fig:Gantt Chart}
\end{figure}
