\section{Time Schedule and Planned Publications}
\label{sec:time_schedule}

% \marginpar{
% \begin{itemize}
%   \item \emph{How can each topic be subdivided into meaningful tasks?}
% 	\item \emph{What is the time schedule?}
% 	\item \emph{Which publications are planned (list of planned publications)?}
% \end{itemize}
% }

\subsection{Planned Publications}
\paragraph{Stochastic Control for Reactive Whole-Body Manipulation} We demonstrate the applicability of our sampling-based method to whole-body coordination for tasks that require multiple contact and contact-switching at the end effector interface. We provide an optimized open-source implementation which can be used to solve a general set of robotic tasks. We introduce system and algorithmic adaptations that enable us to successfully demonstrate the approach for a cabinet door opening task on the real robot.

\paragraph{Affordance-based Control} We investigate how to use affordances for control. Following the research thread about dense affordance predictions, we use these to create a cost map to be used by the sampling-based controller to drive the manipulator towards interaction hotspots, thus removing the need for highly engineered cost functions and interaction priors. We use time to completion and success rate under different starting conditions and task complexities (e.g complex handle types, occlusions) as metrics to evaluate the approach in simulation and on the real system. 

\paragraph{Interactive Affordances} We enrich pure geometrical affordances with wrench information. We will develop an interaction scheme that extract affordances from objects, namely tells where on the object the robot can successfully interact and the orientation/wrench to engage interaction. The data can be generated per object model, before interaction through abstraction in simulation. The affordance map can then be used leveraging the previous work on affordance-based control. The method will be evaluated on a data-set of various objects. 

\paragraph{Interactive Perception} Building on our previous work in integrated perception and interaction planning we will integrate off-the-shelf methods to get a initial coarse estimate of the articulated model and its parameters. A multi-scenario approach will be used to plan under uncertainty and solve the dual problem of estimating the model parameters while achieving the manipulation objective.  

\medskip
The detailed time plan is shown in the Gantt chart in \cref{fig:Gantt Chart}.

% http://www.ctan.org/pkg/pgfgantt

\begin{figure}[]
\begin{center}

\begin{ganttchart}[
x unit = 0.45cm,
y unit title=0.4cm,
y unit chart=0.45cm,
vgrid,
hgrid,
title label anchor/.style={below=-1.7ex},
title height=1,
bar/.style={fill=gray!50,
            draw=black,
            line width=0.5pt},
incomplete/.style={fill=white},
progress label text={},
bar height=0.7,
group left shift=0.0,
group right shift=0.0,
group top shift=0.5,
group height=0.2,
group peaks width=0.4,
group peaks height=0.25,
group peaks tip position=0,
milestone height=0.7,
milestone label anchor/.style={below left=-1.7ex and 0.2ex}
]{1}{24}

% Labels
\gantttitle{Quarter}{24} \\
\gantttitlelist{1,...,12}{2} \\

% Group 1
\ganttgroup{Sampling Control}{1}{4} \\
\ganttbar{Task 1}{1}{2} \\
\ganttbar{Task 2}{2}{3} \\
\ganttbar{Task 3}{2}{4} \\
\ganttbar{Task 4}{2}{4} \\
\ganttmilestone{Publication 1}{5} \\

% Group 2
\ganttnewline
\ganttgroup{Affordance}{5}{8} \\
\ganttbar{Task 5}{5}{6} \\
\ganttbar{Task 6}{6}{8} \\
\ganttbar{Task 7}{6}{8} \\
\ganttmilestone{Publication 2}{9} \\


% Relations
\ganttlink{elem3}{elem7}

\end{ganttchart}
\end{center}
\caption{Gantt Chart}
\label{fig:Gantt Chart}
\end{figure}
