\section{Conclusions} \label{sec:conclusions}

In this work, we have presented a novel control framework for interaction control. We have shown its applicability to the real platform and enhanced its robustness adding new algorithmic components which deploys ZBFs and passivity. We devised a cascaded control architecture which takes into account real-time constraints and applied for the first time these methods on a mobile manipulation platform in real-world experiments. The proposed solution is robust and guarantees stability and safety in case of unforeseen and sudden safety-critical events as we have demonstrated in our simulated and hardware experiments. Last but not least, we open source an efficient multi-threaded implementation of the algorithms that we hope can help practitioners to extend and use this method on new applications.


%Nevertheless there are still some interesting research directions that we aim to address in the near future. As any model-based control method, this is brittle to model mismatches. While stability is guaranteed through passivity, one could exploit interaction to find a better model estimate, closing the loop between perception and control. To this end, the haptic information readily available throughout the simulated rollouts could be actively used to drive an exploratory behavior. Another interesting idea is to use a policy parameterization other than a multivariate Gaussian such that sampling can be done more efficiently in a lower dimensional space. The proposed method is flexible enough to be applied to systems other than mobile manipulators. We hope that the released library will help in this process and with development of future applications. 